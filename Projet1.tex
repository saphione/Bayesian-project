% Options for packages loaded elsewhere
\PassOptionsToPackage{unicode}{hyperref}
\PassOptionsToPackage{hyphens}{url}
%
\documentclass[
]{article}
\usepackage{amsmath,amssymb}
\usepackage{iftex}
\ifPDFTeX
  \usepackage[T1]{fontenc}
  \usepackage[utf8]{inputenc}
  \usepackage{textcomp} % provide euro and other symbols
\else % if luatex or xetex
  \usepackage{unicode-math} % this also loads fontspec
  \defaultfontfeatures{Scale=MatchLowercase}
  \defaultfontfeatures[\rmfamily]{Ligatures=TeX,Scale=1}
\fi
\usepackage{lmodern}
\ifPDFTeX\else
  % xetex/luatex font selection
\fi
% Use upquote if available, for straight quotes in verbatim environments
\IfFileExists{upquote.sty}{\usepackage{upquote}}{}
\IfFileExists{microtype.sty}{% use microtype if available
  \usepackage[]{microtype}
  \UseMicrotypeSet[protrusion]{basicmath} % disable protrusion for tt fonts
}{}
\makeatletter
\@ifundefined{KOMAClassName}{% if non-KOMA class
  \IfFileExists{parskip.sty}{%
    \usepackage{parskip}
  }{% else
    \setlength{\parindent}{0pt}
    \setlength{\parskip}{6pt plus 2pt minus 1pt}}
}{% if KOMA class
  \KOMAoptions{parskip=half}}
\makeatother
\usepackage{xcolor}
\usepackage[margin=1in]{geometry}
\usepackage{color}
\usepackage{fancyvrb}
\newcommand{\VerbBar}{|}
\newcommand{\VERB}{\Verb[commandchars=\\\{\}]}
\DefineVerbatimEnvironment{Highlighting}{Verbatim}{commandchars=\\\{\}}
% Add ',fontsize=\small' for more characters per line
\usepackage{framed}
\definecolor{shadecolor}{RGB}{248,248,248}
\newenvironment{Shaded}{\begin{snugshade}}{\end{snugshade}}
\newcommand{\AlertTok}[1]{\textcolor[rgb]{0.94,0.16,0.16}{#1}}
\newcommand{\AnnotationTok}[1]{\textcolor[rgb]{0.56,0.35,0.01}{\textbf{\textit{#1}}}}
\newcommand{\AttributeTok}[1]{\textcolor[rgb]{0.13,0.29,0.53}{#1}}
\newcommand{\BaseNTok}[1]{\textcolor[rgb]{0.00,0.00,0.81}{#1}}
\newcommand{\BuiltInTok}[1]{#1}
\newcommand{\CharTok}[1]{\textcolor[rgb]{0.31,0.60,0.02}{#1}}
\newcommand{\CommentTok}[1]{\textcolor[rgb]{0.56,0.35,0.01}{\textit{#1}}}
\newcommand{\CommentVarTok}[1]{\textcolor[rgb]{0.56,0.35,0.01}{\textbf{\textit{#1}}}}
\newcommand{\ConstantTok}[1]{\textcolor[rgb]{0.56,0.35,0.01}{#1}}
\newcommand{\ControlFlowTok}[1]{\textcolor[rgb]{0.13,0.29,0.53}{\textbf{#1}}}
\newcommand{\DataTypeTok}[1]{\textcolor[rgb]{0.13,0.29,0.53}{#1}}
\newcommand{\DecValTok}[1]{\textcolor[rgb]{0.00,0.00,0.81}{#1}}
\newcommand{\DocumentationTok}[1]{\textcolor[rgb]{0.56,0.35,0.01}{\textbf{\textit{#1}}}}
\newcommand{\ErrorTok}[1]{\textcolor[rgb]{0.64,0.00,0.00}{\textbf{#1}}}
\newcommand{\ExtensionTok}[1]{#1}
\newcommand{\FloatTok}[1]{\textcolor[rgb]{0.00,0.00,0.81}{#1}}
\newcommand{\FunctionTok}[1]{\textcolor[rgb]{0.13,0.29,0.53}{\textbf{#1}}}
\newcommand{\ImportTok}[1]{#1}
\newcommand{\InformationTok}[1]{\textcolor[rgb]{0.56,0.35,0.01}{\textbf{\textit{#1}}}}
\newcommand{\KeywordTok}[1]{\textcolor[rgb]{0.13,0.29,0.53}{\textbf{#1}}}
\newcommand{\NormalTok}[1]{#1}
\newcommand{\OperatorTok}[1]{\textcolor[rgb]{0.81,0.36,0.00}{\textbf{#1}}}
\newcommand{\OtherTok}[1]{\textcolor[rgb]{0.56,0.35,0.01}{#1}}
\newcommand{\PreprocessorTok}[1]{\textcolor[rgb]{0.56,0.35,0.01}{\textit{#1}}}
\newcommand{\RegionMarkerTok}[1]{#1}
\newcommand{\SpecialCharTok}[1]{\textcolor[rgb]{0.81,0.36,0.00}{\textbf{#1}}}
\newcommand{\SpecialStringTok}[1]{\textcolor[rgb]{0.31,0.60,0.02}{#1}}
\newcommand{\StringTok}[1]{\textcolor[rgb]{0.31,0.60,0.02}{#1}}
\newcommand{\VariableTok}[1]{\textcolor[rgb]{0.00,0.00,0.00}{#1}}
\newcommand{\VerbatimStringTok}[1]{\textcolor[rgb]{0.31,0.60,0.02}{#1}}
\newcommand{\WarningTok}[1]{\textcolor[rgb]{0.56,0.35,0.01}{\textbf{\textit{#1}}}}
\usepackage{graphicx}
\makeatletter
\def\maxwidth{\ifdim\Gin@nat@width>\linewidth\linewidth\else\Gin@nat@width\fi}
\def\maxheight{\ifdim\Gin@nat@height>\textheight\textheight\else\Gin@nat@height\fi}
\makeatother
% Scale images if necessary, so that they will not overflow the page
% margins by default, and it is still possible to overwrite the defaults
% using explicit options in \includegraphics[width, height, ...]{}
\setkeys{Gin}{width=\maxwidth,height=\maxheight,keepaspectratio}
% Set default figure placement to htbp
\makeatletter
\def\fps@figure{htbp}
\makeatother
\setlength{\emergencystretch}{3em} % prevent overfull lines
\providecommand{\tightlist}{%
  \setlength{\itemsep}{0pt}\setlength{\parskip}{0pt}}
\setcounter{secnumdepth}{-\maxdimen} % remove section numbering
\ifLuaTeX
  \usepackage{selnolig}  % disable illegal ligatures
\fi
\usepackage{bookmark}
\IfFileExists{xurl.sty}{\usepackage{xurl}}{} % add URL line breaks if available
\urlstyle{same}
\hypersetup{
  pdftitle={Projet1\_Bayesian},
  pdfauthor={Lorazo, M. De Koninck, A},
  hidelinks,
  pdfcreator={LaTeX via pandoc}}

\title{Projet1\_Bayesian}
\author{Lorazo, M. De Koninck, A}
\date{2024-12-07}

\begin{document}
\maketitle

\section{Bayesian's project}\label{bayesians-project}

Article : Patches of Bare Ground as a Staple Commodity for Declining
Ground-Foraging Insectivorous Farmland Birds

\subsection{1. Loading and exploring the
data}\label{loading-and-exploring-the-data}

\subsubsection{Loading packages}\label{loading-packages}

\subsubsection{Loading dataset}\label{loading-dataset}

\subsubsection{What about the data ??}\label{what-about-the-data}

\paragraph{Creation of the explicative
variables}\label{creation-of-the-explicative-variables}

We don't forget to scale the variables.

We put the response variable as a numerical variable instead of a
character one.

Doesn't seem to have a relation between the y variable and the two
explicative variables ??

\paragraph{Classic representations}\label{classic-representations}

\begin{Shaded}
\begin{Highlighting}[]
\NormalTok{hoopoe}\SpecialCharTok{$}\NormalTok{y }\OtherTok{\textless{}{-}} \FunctionTok{ifelse}\NormalTok{(hoopoe}\SpecialCharTok{$}\NormalTok{y }\SpecialCharTok{==} \StringTok{"random"}\NormalTok{, }\DecValTok{0}\NormalTok{, }\DecValTok{1}\NormalTok{)}
\FunctionTok{hist}\NormalTok{(hoopoe}\SpecialCharTok{$}\NormalTok{bare)}
\end{Highlighting}
\end{Shaded}

\includegraphics{Projet1_files/figure-latex/Some classic representation-1.pdf}

\begin{Shaded}
\begin{Highlighting}[]
\FunctionTok{hist}\NormalTok{(hoopoe}\SpecialCharTok{$}\NormalTok{heightC)}
\end{Highlighting}
\end{Shaded}

\includegraphics{Projet1_files/figure-latex/Some classic representation-2.pdf}

\begin{Shaded}
\begin{Highlighting}[]
\FunctionTok{plot}\NormalTok{(hoopoe}\SpecialCharTok{$}\NormalTok{bare, hoopoe}\SpecialCharTok{$}\NormalTok{y)}
\end{Highlighting}
\end{Shaded}

\includegraphics{Projet1_files/figure-latex/Some classic representation-3.pdf}

\begin{Shaded}
\begin{Highlighting}[]
\FunctionTok{plot}\NormalTok{(hoopoe}\SpecialCharTok{$}\NormalTok{heightC, hoopoe}\SpecialCharTok{$}\NormalTok{y)}
\end{Highlighting}
\end{Shaded}

\includegraphics{Projet1_files/figure-latex/Some classic representation-4.pdf}

\subsection{2. Ajusting a simple model}\label{ajusting-a-simple-model}

Our goal is to find a relation between the \%bareground or the
vegetation height and the response variable.

For that, we will use a Bayesian approach (since that the name of the
course lol) while considering random effects on the slope and on the
intercept. To construct that model, we need priors and data. Since we
don't have any assumption available on the two explicative variables
(actually yes we can imagine some with the representation but let's
forget about that), we will consider a normal distribution for the
priors, which are here not informative. So, one of the assumptions is
that we have enough information in our data to represent the structure
and the interaction that we are researching (annnnnnd we have 13
individuals\ldots. robust analysis).

\subsubsection{Construction of the
model}\label{construction-of-the-model}

To construct the model, we need to regroup a lot of variables inside an
only dataset.

The model take two things into account : the number of rows (N), the
number of observations, and the number of individuals (nb), what we will
consider as a possible random effet. Our model have one response
variable (y), which is boolean (0 if the considered random loaction is
random, 1 if it is not) and four explicative variables (\%bareground,
square(\%bareground), vegetation height, square(vegetation height).

\subsubsection{Creation of the dataset}\label{creation-of-the-dataset}

\begin{Shaded}
\begin{Highlighting}[]
\NormalTok{datahoopoe }\OtherTok{\textless{}{-}} \FunctionTok{list}\NormalTok{(}
  \AttributeTok{N =} \FunctionTok{length}\NormalTok{(hoopoe}\SpecialCharTok{$}\NormalTok{y),}
  \AttributeTok{y =} \FunctionTok{ifelse}\NormalTok{(hoopoe}\SpecialCharTok{$}\NormalTok{y }\SpecialCharTok{==} \StringTok{"random"}\NormalTok{, }\DecValTok{0}\NormalTok{, }\DecValTok{1}\NormalTok{),}
  \AttributeTok{ring =} \FunctionTok{as.numeric}\NormalTok{(hoopoe}\SpecialCharTok{$}\NormalTok{ring),}
  \AttributeTok{nb =} \FunctionTok{length}\NormalTok{(}\FunctionTok{levels}\NormalTok{(hoopoe}\SpecialCharTok{$}\NormalTok{ring)),}
  \AttributeTok{bare =} \FunctionTok{as.numeric}\NormalTok{(}\FunctionTok{scale}\NormalTok{(hoopoe}\SpecialCharTok{$}\NormalTok{bare)),}
  \AttributeTok{bare\_carre =} \FunctionTok{as.numeric}\NormalTok{(}\FunctionTok{scale}\NormalTok{(hoopoe}\SpecialCharTok{$}\NormalTok{x\_bg\_carr)),}
  \AttributeTok{height =} \FunctionTok{as.numeric}\NormalTok{(}\FunctionTok{scale}\NormalTok{(hoopoe}\SpecialCharTok{$}\NormalTok{heightC)),}
  \AttributeTok{height\_carre =} \FunctionTok{as.numeric}\NormalTok{(}\FunctionTok{scale}\NormalTok{(hoopoe}\SpecialCharTok{$}\NormalTok{x\_vh\_carr))}
\NormalTok{)}
\end{Highlighting}
\end{Shaded}

\subsubsection{Construction of the
function}\label{construction-of-the-function}

To run the MCMC method, we need a function that will be put in input on
Jags. That function takes into account random effect on the slopes and
on the intercept. We then input the initial values of the parameters. We
can finally run the model.

\begin{Shaded}
\begin{Highlighting}[]
\CommentTok{\# Function}
\NormalTok{logisitic }\OtherTok{\textless{}{-}} \ControlFlowTok{function}\NormalTok{()\{}
  \ControlFlowTok{for}\NormalTok{(i }\ControlFlowTok{in} \DecValTok{1}\SpecialCharTok{:}\NormalTok{N)\{}
\NormalTok{    y[i] }\SpecialCharTok{\textasciitilde{}} \FunctionTok{dbin}\NormalTok{(p[i],}\DecValTok{1}\NormalTok{) }\CommentTok{\# Binomial likelihood}
    \FunctionTok{logit}\NormalTok{(p[i]) }\OtherTok{\textless{}{-}}\NormalTok{ a[ring[i]] }\SpecialCharTok{+}\NormalTok{ b.bare[ring[i]] }\SpecialCharTok{*}\NormalTok{ bare[i] }\SpecialCharTok{+}\NormalTok{ b.bare\_carre[ring[i]] }\SpecialCharTok{*}\NormalTok{ bare\_carre[i] }\SpecialCharTok{+}\NormalTok{ b.height[ring[i]] }\SpecialCharTok{*}\NormalTok{ height[i] }\SpecialCharTok{+}\NormalTok{ b.height\_carre[ring[i]] }\SpecialCharTok{*}\NormalTok{ height\_carre[i]}
\NormalTok{  \}}
  \ControlFlowTok{for}\NormalTok{ (j }\ControlFlowTok{in} \DecValTok{1}\SpecialCharTok{:}\NormalTok{nb)\{}
\NormalTok{    a[j] }\SpecialCharTok{\textasciitilde{}} \FunctionTok{dnorm}\NormalTok{(mua,tau.a) }
\NormalTok{    b.bare[j] }\SpecialCharTok{\textasciitilde{}} \FunctionTok{dnorm}\NormalTok{(mubare,tau.b)}
\NormalTok{    b.bare\_carre[j] }\SpecialCharTok{\textasciitilde{}} \FunctionTok{dnorm}\NormalTok{(mubarec,tau.bc)}
\NormalTok{    b.height[j] }\SpecialCharTok{\textasciitilde{}} \FunctionTok{dnorm}\NormalTok{(muheight,tau.v)}
\NormalTok{    b.height\_carre[j] }\SpecialCharTok{\textasciitilde{}} \FunctionTok{dnorm}\NormalTok{(muheightc,tau.vc) }
\NormalTok{  \}}
\NormalTok{mua }\SpecialCharTok{\textasciitilde{}} \FunctionTok{dnorm}\NormalTok{(}\DecValTok{0}\NormalTok{,}\FloatTok{0.001}\NormalTok{) }
\NormalTok{mubare }\SpecialCharTok{\textasciitilde{}} \FunctionTok{dnorm}\NormalTok{(}\DecValTok{0}\NormalTok{,}\FloatTok{0.001}\NormalTok{) }
\NormalTok{mubarec }\SpecialCharTok{\textasciitilde{}} \FunctionTok{dnorm}\NormalTok{(}\DecValTok{0}\NormalTok{,}\FloatTok{0.001}\NormalTok{) }
\NormalTok{muheight }\SpecialCharTok{\textasciitilde{}} \FunctionTok{dnorm}\NormalTok{(}\DecValTok{0}\NormalTok{,}\FloatTok{0.001}\NormalTok{)}
\NormalTok{muheightc }\SpecialCharTok{\textasciitilde{}} \FunctionTok{dnorm}\NormalTok{(}\DecValTok{0}\NormalTok{,}\FloatTok{0.001}\NormalTok{)}
\NormalTok{tau.a }\OtherTok{\textless{}{-}} \DecValTok{1} \SpecialCharTok{/}\NormalTok{ (sigma.a }\SpecialCharTok{*}\NormalTok{ sigma.a)}
\NormalTok{sigma.a }\SpecialCharTok{\textasciitilde{}} \FunctionTok{dunif}\NormalTok{(}\DecValTok{0}\NormalTok{,}\DecValTok{100}\NormalTok{)}
\NormalTok{tau.b }\OtherTok{\textless{}{-}} \DecValTok{1} \SpecialCharTok{/}\NormalTok{ (sigma.b }\SpecialCharTok{*}\NormalTok{ sigma.b)}
\NormalTok{sigma.b }\SpecialCharTok{\textasciitilde{}} \FunctionTok{dunif}\NormalTok{(}\DecValTok{0}\NormalTok{,}\DecValTok{100}\NormalTok{)}
\NormalTok{tau.bc }\OtherTok{\textless{}{-}} \DecValTok{1} \SpecialCharTok{/}\NormalTok{ (sigma.bc }\SpecialCharTok{*}\NormalTok{ sigma.bc)}
\NormalTok{sigma.bc }\SpecialCharTok{\textasciitilde{}} \FunctionTok{dunif}\NormalTok{(}\DecValTok{0}\NormalTok{,}\DecValTok{100}\NormalTok{)}
\NormalTok{tau.v }\OtherTok{\textless{}{-}} \DecValTok{1} \SpecialCharTok{/}\NormalTok{ (sigma.v }\SpecialCharTok{*}\NormalTok{ sigma.v)}
\NormalTok{sigma.v }\SpecialCharTok{\textasciitilde{}} \FunctionTok{dunif}\NormalTok{(}\DecValTok{0}\NormalTok{,}\DecValTok{100}\NormalTok{)}
\NormalTok{tau.vc }\OtherTok{\textless{}{-}} \DecValTok{1} \SpecialCharTok{/}\NormalTok{ (sigma.vc }\SpecialCharTok{*}\NormalTok{ sigma.vc)}
\NormalTok{sigma.vc }\SpecialCharTok{\textasciitilde{}} \FunctionTok{dunif}\NormalTok{(}\DecValTok{0}\NormalTok{,}\DecValTok{100}\NormalTok{)}
\NormalTok{\}}

\CommentTok{\# Initial values}
\NormalTok{init1 }\OtherTok{\textless{}{-}} \FunctionTok{list}\NormalTok{(}\AttributeTok{a =} \FunctionTok{rep}\NormalTok{(}\FloatTok{0.5}\NormalTok{, datahoopoe}\SpecialCharTok{$}\NormalTok{nb), }\AttributeTok{b.bare =} \FunctionTok{rep}\NormalTok{(}\DecValTok{1}\NormalTok{, datahoopoe}\SpecialCharTok{$}\NormalTok{nb), }\AttributeTok{b.bare\_carre =} \FunctionTok{rep}\NormalTok{(}\DecValTok{1}\NormalTok{, datahoopoe}\SpecialCharTok{$}\NormalTok{nb), }\AttributeTok{b.height =} \FunctionTok{rep}\NormalTok{(}\DecValTok{1}\NormalTok{, datahoopoe}\SpecialCharTok{$}\NormalTok{nb), }\AttributeTok{b.height\_carre =} \FunctionTok{rep}\NormalTok{(}\DecValTok{1}\NormalTok{, datahoopoe}\SpecialCharTok{$}\NormalTok{nb))}
\NormalTok{init2 }\OtherTok{\textless{}{-}} \FunctionTok{list}\NormalTok{(}\AttributeTok{a =} \FunctionTok{rep}\NormalTok{(}\SpecialCharTok{{-}}\FloatTok{0.5}\NormalTok{, datahoopoe}\SpecialCharTok{$}\NormalTok{nb), }\AttributeTok{b.bare =} \FunctionTok{rep}\NormalTok{(}\SpecialCharTok{{-}}\DecValTok{1}\NormalTok{, datahoopoe}\SpecialCharTok{$}\NormalTok{nb), }\AttributeTok{b.bare\_carre =} \FunctionTok{rep}\NormalTok{(}\SpecialCharTok{{-}}\DecValTok{1}\NormalTok{, datahoopoe}\SpecialCharTok{$}\NormalTok{nb), }\AttributeTok{b.height =} \FunctionTok{rep}\NormalTok{(}\SpecialCharTok{{-}}\DecValTok{1}\NormalTok{, datahoopoe}\SpecialCharTok{$}\NormalTok{nb), }\AttributeTok{b.height\_carre =} \FunctionTok{rep}\NormalTok{(}\SpecialCharTok{{-}}\DecValTok{1}\NormalTok{, datahoopoe}\SpecialCharTok{$}\NormalTok{nb))}
\NormalTok{init3 }\OtherTok{\textless{}{-}} \FunctionTok{list}\NormalTok{(}\AttributeTok{a =} \FunctionTok{rep}\NormalTok{(}\DecValTok{0}\NormalTok{, datahoopoe}\SpecialCharTok{$}\NormalTok{nb), }\AttributeTok{b.bare =} \FunctionTok{rep}\NormalTok{(}\DecValTok{0}\NormalTok{, datahoopoe}\SpecialCharTok{$}\NormalTok{nb), }\AttributeTok{b.bare\_carre =} \FunctionTok{rep}\NormalTok{(}\DecValTok{0}\NormalTok{, datahoopoe}\SpecialCharTok{$}\NormalTok{nb), }\AttributeTok{b.height =} \FunctionTok{rep}\NormalTok{(}\DecValTok{0}\NormalTok{, datahoopoe}\SpecialCharTok{$}\NormalTok{nb), }\AttributeTok{b.height\_carre =} \FunctionTok{rep}\NormalTok{(}\DecValTok{0}\NormalTok{, datahoopoe}\SpecialCharTok{$}\NormalTok{nb))}
\NormalTok{init4 }\OtherTok{\textless{}{-}} \FunctionTok{list}\NormalTok{(}\AttributeTok{a =} \FunctionTok{rep}\NormalTok{(}\SpecialCharTok{{-}}\DecValTok{1}\NormalTok{, datahoopoe}\SpecialCharTok{$}\NormalTok{nb), }\AttributeTok{b.bare =} \FunctionTok{rep}\NormalTok{(}\SpecialCharTok{{-}}\DecValTok{1}\NormalTok{, datahoopoe}\SpecialCharTok{$}\NormalTok{nb), }\AttributeTok{b.bare\_carre =} \FunctionTok{rep}\NormalTok{(}\SpecialCharTok{{-}}\DecValTok{1}\NormalTok{, datahoopoe}\SpecialCharTok{$}\NormalTok{nb), }\AttributeTok{b.height =} \FunctionTok{rep}\NormalTok{(}\SpecialCharTok{{-}}\DecValTok{1}\NormalTok{, datahoopoe}\SpecialCharTok{$}\NormalTok{nb), }\AttributeTok{b.height\_carre =} \FunctionTok{rep}\NormalTok{(}\SpecialCharTok{{-}}\DecValTok{1}\NormalTok{, datahoopoe}\SpecialCharTok{$}\NormalTok{nb))}
\NormalTok{init5 }\OtherTok{\textless{}{-}} \FunctionTok{list}\NormalTok{(}\AttributeTok{a =} \FunctionTok{rep}\NormalTok{(}\DecValTok{1}\NormalTok{, datahoopoe}\SpecialCharTok{$}\NormalTok{nb), }\AttributeTok{b.bare =} \FunctionTok{rep}\NormalTok{(}\DecValTok{1}\NormalTok{, datahoopoe}\SpecialCharTok{$}\NormalTok{nb), }\AttributeTok{b.bare\_carre =} \FunctionTok{rep}\NormalTok{(}\DecValTok{1}\NormalTok{, datahoopoe}\SpecialCharTok{$}\NormalTok{nb), }\AttributeTok{b.height =} \FunctionTok{rep}\NormalTok{(}\DecValTok{1}\NormalTok{, datahoopoe}\SpecialCharTok{$}\NormalTok{nb), }\AttributeTok{b.height\_carre =} \FunctionTok{rep}\NormalTok{(}\DecValTok{1}\NormalTok{, datahoopoe}\SpecialCharTok{$}\NormalTok{nb))}
\NormalTok{init }\OtherTok{\textless{}{-}} \FunctionTok{list}\NormalTok{(init1, init2, init3, init4, init5)}

\CommentTok{\# Parameters to monitor}
\NormalTok{params }\OtherTok{\textless{}{-}} \FunctionTok{c}\NormalTok{(}\StringTok{"a"}\NormalTok{,}\StringTok{"b.bare"}\NormalTok{, }\StringTok{"b.bare\_carre"}\NormalTok{, }\StringTok{"b.height"}\NormalTok{, }\StringTok{"b.height\_carre"}\NormalTok{)}

\CommentTok{\# Call jags to fit model}
\NormalTok{partial\_pooling\_fit }\OtherTok{\textless{}{-}} \FunctionTok{jags}\NormalTok{(}\AttributeTok{data =}\NormalTok{ datahoopoe,}
                             \AttributeTok{inits =}\NormalTok{ init,}
                             \AttributeTok{parameters.to.save =}\NormalTok{ params,}
                             \AttributeTok{model.file =}\NormalTok{ logisitic,}
                             \AttributeTok{n.chains =} \DecValTok{5}\NormalTok{,}
                             \AttributeTok{n.iter =} \DecValTok{60000}\NormalTok{,}
                             \AttributeTok{n.burnin =} \DecValTok{15000}\NormalTok{,}
                             \AttributeTok{n.thin =} \DecValTok{1}\NormalTok{)}
\end{Highlighting}
\end{Shaded}

\begin{verbatim}
## module glm loaded
\end{verbatim}

\begin{verbatim}
## Compiling model graph
##    Resolving undeclared variables
##    Allocating nodes
## Graph information:
##    Observed stochastic nodes: 1048
##    Unobserved stochastic nodes: 75
##    Total graph size: 8061
## 
## Initializing model
\end{verbatim}

\begin{Shaded}
\begin{Highlighting}[]
\NormalTok{partial\_pooling\_fit}
\end{Highlighting}
\end{Shaded}

\begin{verbatim}
## Inference for Bugs model at "C:/Users/Saphione/AppData/Local/Temp/RtmpOOsBkl/model154867217ed6", fit using jags,
##  5 chains, each with 60000 iterations (first 15000 discarded)
##  n.sims = 225000 iterations saved. Running time = 293.57 secs
##                    mu.vect sd.vect    2.5%     25%     50%    75%   97.5%  Rhat
## a[1]                56.298  13.682  31.354  47.519  55.358 64.579  89.450 1.356
## a[2]                58.682  14.594  33.518  48.738  57.260 67.471  91.770 1.233
## a[3]                60.688  13.768  38.091  51.231  58.783 68.774  92.017 1.360
## a[4]                60.076  16.151  31.842  49.568  58.379 69.392  96.281 1.177
## a[5]                57.036  12.822  34.399  48.202  56.612 64.800  85.933 1.208
## a[6]                59.478  15.173  33.970  49.204  57.817 68.169  93.793 1.296
## a[7]                62.958  17.500  35.133  50.181  60.220 73.851 102.084 1.468
## a[8]                62.715  16.991  36.346  51.191  59.397 71.669 103.394 1.240
## a[9]                60.136  13.442  36.511  51.161  58.751 67.973  91.592 1.256
## a[10]               61.244  14.882  33.912  51.358  59.429 71.145  91.665 1.350
## a[11]               60.832  18.025  34.571  49.451  57.934 68.758 102.074 1.239
## a[12]               63.046  17.404  36.702  51.278  58.979 71.897 103.974 1.431
## a[13]               60.782  14.732  36.120  50.792  58.878 69.393  94.068 1.138
## b.bare[1]          -10.894  18.851 -54.001 -22.125 -10.384  2.009  22.359 1.091
## b.bare[2]          -10.655  17.902 -50.301 -21.772  -9.714  1.176  21.307 1.082
## b.bare[3]          -11.075  18.437 -52.038 -22.879  -9.439  1.909  20.143 1.066
## b.bare[4]           -9.750  18.523 -53.193 -20.715  -8.753  2.326  23.307 1.070
## b.bare[5]          -11.600  18.727 -54.714 -22.467 -10.322  0.621  22.407 1.089
## b.bare[6]          -11.097  18.068 -50.394 -22.416  -9.944  1.401  20.135 1.056
## b.bare[7]          -10.071  17.482 -49.402 -20.844  -9.271  1.715  20.970 1.047
## b.bare[8]          -10.277  18.213 -51.163 -21.037  -8.637  1.861  21.139 1.065
## b.bare[9]          -11.620  17.913 -55.435 -22.375 -10.495  0.457  18.727 1.073
## b.bare[10]         -10.837  17.791 -51.461 -21.717  -8.438  1.392  18.563 1.110
## b.bare[11]         -11.030  18.311 -54.340 -21.839  -9.613  1.316  20.371 1.061
## b.bare[12]         -11.688  17.547 -52.536 -23.055  -9.865  0.416  17.841 1.071
## b.bare[13]         -11.577  18.343 -54.660 -22.834 -10.286  0.943  20.965 1.086
## b.bare_carre[1]     14.999  17.258 -15.417   2.844  13.895 26.303  52.227 1.133
## b.bare_carre[2]     15.496  19.141 -16.040   2.395  13.554 26.430  58.274 1.079
## b.bare_carre[3]     14.944  18.031 -15.008   2.505  13.095 25.310  56.485 1.099
## b.bare_carre[4]     15.948  18.083 -14.645   3.209  14.452 27.256  55.357 1.111
## b.bare_carre[5]     15.051  18.176 -16.113   2.183  13.645 26.888  54.152 1.095
## b.bare_carre[6]     14.261  18.832 -20.896   2.285  13.466 25.827  54.202 1.141
## b.bare_carre[7]     16.135  19.151 -14.620   2.846  13.465 27.190  61.479 1.186
## b.bare_carre[8]     14.990  18.053 -15.328   2.109  13.514 26.214  54.323 1.142
## b.bare_carre[9]     14.622  17.917 -15.737   1.873  13.203 25.346  55.032 1.129
## b.bare_carre[10]    14.710  17.120 -13.796   2.636  12.910 24.954  53.367 1.130
## b.bare_carre[11]    15.619  18.967 -15.464   2.246  13.617 27.013  58.238 1.085
## b.bare_carre[12]    14.565  18.577 -15.057   1.522  12.563 25.216  57.223 1.130
## b.bare_carre[13]    14.608  16.111 -13.217   3.657  13.381 23.648  52.587 1.120
## b.height[1]         -5.901  20.946 -47.427 -19.407  -6.546  6.530  38.218 1.093
## b.height[2]         -6.800  21.658 -48.656 -21.401  -7.148  7.113  36.603 1.082
## b.height[3]         -6.807  21.046 -46.307 -21.076  -7.891  5.709  38.932 1.118
## b.height[4]         -6.460  21.051 -47.303 -20.593  -6.899  6.669  36.282 1.086
## b.height[5]         -6.567  21.073 -48.545 -20.361  -6.681  6.891  34.624 1.078
## b.height[6]         -6.703  20.337 -47.694 -20.178  -6.386  6.009  34.478 1.098
## b.height[7]         -5.009  22.199 -47.599 -19.096  -6.423  7.741  42.068 1.110
## b.height[8]         -6.634  19.655 -46.362 -19.541  -6.314  5.865  32.829 1.063
## b.height[9]         -6.150  20.550 -47.320 -20.041  -6.010  7.446  33.381 1.061
## b.height[10]        -6.607  20.321 -46.603 -19.649  -6.926  5.521  35.143 1.102
## b.height[11]        -6.045  20.999 -47.786 -19.522  -6.423  6.567  37.237 1.106
## b.height[12]        -5.723  20.766 -46.640 -19.597  -6.360  6.968  35.844 1.087
## b.height[13]        -7.502  21.037 -49.470 -21.344  -8.001  5.395  35.631 1.103
## b.height_carre[1]   14.814  25.193 -35.295  -2.052  15.670 31.015  64.785 1.315
## b.height_carre[2]   13.673  24.464 -33.467  -3.282  14.323 29.918  62.787 1.301
## b.height_carre[3]   13.608  24.184 -31.902  -3.123  13.813 29.415  65.131 1.203
## b.height_carre[4]   13.243  25.006 -35.343  -3.677  14.112 29.815  63.350 1.257
## b.height_carre[5]   13.551  24.285 -32.475  -3.669  14.266 30.090  62.001 1.254
## b.height_carre[6]   13.501  24.906 -34.328  -3.142  14.659 29.436  61.280 1.247
## b.height_carre[7]   14.624  23.852 -31.559  -1.907  15.041 30.467  62.686 1.277
## b.height_carre[8]   13.789  25.245 -35.756  -2.966  15.352 29.862  64.497 1.298
## b.height_carre[9]   13.773  26.287 -36.861  -3.789  14.382 31.308  65.334 1.364
## b.height_carre[10]  13.802  24.962 -35.340  -2.883  15.133 29.628  64.266 1.278
## b.height_carre[11]  13.755  24.727 -33.622  -3.171  14.622 30.044  63.267 1.195
## b.height_carre[12]  14.106  24.477 -32.102  -3.071  14.800 30.112  65.940 1.329
## b.height_carre[13]  13.729  24.796 -33.355  -3.541  14.537 30.291  63.674 1.258
## deviance             0.241   0.743   0.000   0.000   0.002  0.089   2.364 1.018
##                    n.eff
## a[1]                  13
## a[2]                  19
## a[3]                  13
## a[4]                  24
## a[5]                  20
## a[6]                  15
## a[7]                  12
## a[8]                  18
## a[9]                  17
## a[10]                 14
## a[11]                 18
## a[12]                 12
## a[13]                 30
## b.bare[1]             46
## b.bare[2]             47
## b.bare[3]             60
## b.bare[4]             61
## b.bare[5]             42
## b.bare[6]             76
## b.bare[7]             86
## b.bare[8]             78
## b.bare[9]             51
## b.bare[10]            35
## b.bare[11]            69
## b.bare[12]            62
## b.bare[13]            47
## b.bare_carre[1]       28
## b.bare_carre[2]       46
## b.bare_carre[3]       37
## b.bare_carre[4]       33
## b.bare_carre[5]       38
## b.bare_carre[6]       27
## b.bare_carre[7]       22
## b.bare_carre[8]       27
## b.bare_carre[9]       29
## b.bare_carre[10]      30
## b.bare_carre[11]      42
## b.bare_carre[12]      32
## b.bare_carre[13]      32
## b.height[1]           48
## b.height[2]           49
## b.height[3]           35
## b.height[4]           46
## b.height[5]           48
## b.height[6]           39
## b.height[7]           40
## b.height[8]           61
## b.height[9]           64
## b.height[10]          38
## b.height[11]          36
## b.height[12]          46
## b.height[13]          39
## b.height_carre[1]     15
## b.height_carre[2]     15
## b.height_carre[3]     20
## b.height_carre[4]     17
## b.height_carre[5]     17
## b.height_carre[6]     18
## b.height_carre[7]     16
## b.height_carre[8]     15
## b.height_carre[9]     13
## b.height_carre[10]    16
## b.height_carre[11]    21
## b.height_carre[12]    14
## b.height_carre[13]    17
## deviance             440
## 
## For each parameter, n.eff is a crude measure of effective sample size,
## and Rhat is the potential scale reduction factor (at convergence, Rhat=1).
## 
## DIC info (using the rule: pV = var(deviance)/2)
## pV = 0.3 and DIC = 0.5
## DIC is an estimate of expected predictive error (lower deviance is better).
\end{verbatim}

\begin{Shaded}
\begin{Highlighting}[]
\CommentTok{\#par(mfrow = c(1, 2))}
\CommentTok{\#hist(partial\_pooling\_fit$BUGSoutput$sims.matrix)}
\CommentTok{\#plot(density(partial\_pooling\_fit$BUGSoutput$sims.matrix))}
\end{Highlighting}
\end{Shaded}

\subsection{3. Comparing the different
models}\label{comparing-the-different-models}

Sooooo, like, what are we suppose to put here ???

We are gonna use the DIC to choose the best model, which is \ldots..

\subsection{4. Infering and interpreting the best
model}\label{infering-and-interpreting-the-best-model}

Sur la base du meilleur modèle, donnez les estimations des paramètres
ainsi qu'une mesure de l'incertitude associée. Interprétez vos
résultats.

\begin{Shaded}
\begin{Highlighting}[]
\FunctionTok{traceplot}\NormalTok{(partial\_pooling\_fit)}
\end{Highlighting}
\end{Shaded}

\includegraphics{Projet1_files/figure-latex/unnamed-chunk-2-1.pdf}
\includegraphics{Projet1_files/figure-latex/unnamed-chunk-2-2.pdf}
\includegraphics{Projet1_files/figure-latex/unnamed-chunk-2-3.pdf}
\includegraphics{Projet1_files/figure-latex/unnamed-chunk-2-4.pdf}
\includegraphics{Projet1_files/figure-latex/unnamed-chunk-2-5.pdf}
\includegraphics{Projet1_files/figure-latex/unnamed-chunk-2-6.pdf}
\includegraphics{Projet1_files/figure-latex/unnamed-chunk-2-7.pdf}
\includegraphics{Projet1_files/figure-latex/unnamed-chunk-2-8.pdf}
\includegraphics{Projet1_files/figure-latex/unnamed-chunk-2-9.pdf}
\includegraphics{Projet1_files/figure-latex/unnamed-chunk-2-10.pdf}
\includegraphics{Projet1_files/figure-latex/unnamed-chunk-2-11.pdf}
\includegraphics{Projet1_files/figure-latex/unnamed-chunk-2-12.pdf}
\includegraphics{Projet1_files/figure-latex/unnamed-chunk-2-13.pdf}
\includegraphics{Projet1_files/figure-latex/unnamed-chunk-2-14.pdf}
\includegraphics{Projet1_files/figure-latex/unnamed-chunk-2-15.pdf}
\includegraphics{Projet1_files/figure-latex/unnamed-chunk-2-16.pdf}
\includegraphics{Projet1_files/figure-latex/unnamed-chunk-2-17.pdf}
\includegraphics{Projet1_files/figure-latex/unnamed-chunk-2-18.pdf}
\includegraphics{Projet1_files/figure-latex/unnamed-chunk-2-19.pdf}
\includegraphics{Projet1_files/figure-latex/unnamed-chunk-2-20.pdf}
\includegraphics{Projet1_files/figure-latex/unnamed-chunk-2-21.pdf}
\includegraphics{Projet1_files/figure-latex/unnamed-chunk-2-22.pdf}
\includegraphics{Projet1_files/figure-latex/unnamed-chunk-2-23.pdf}
\includegraphics{Projet1_files/figure-latex/unnamed-chunk-2-24.pdf}
\includegraphics{Projet1_files/figure-latex/unnamed-chunk-2-25.pdf}
\includegraphics{Projet1_files/figure-latex/unnamed-chunk-2-26.pdf}
\includegraphics{Projet1_files/figure-latex/unnamed-chunk-2-27.pdf}
\includegraphics{Projet1_files/figure-latex/unnamed-chunk-2-28.pdf}
\includegraphics{Projet1_files/figure-latex/unnamed-chunk-2-29.pdf}
\includegraphics{Projet1_files/figure-latex/unnamed-chunk-2-30.pdf}
\includegraphics{Projet1_files/figure-latex/unnamed-chunk-2-31.pdf}
\includegraphics{Projet1_files/figure-latex/unnamed-chunk-2-32.pdf}
\includegraphics{Projet1_files/figure-latex/unnamed-chunk-2-33.pdf}
\includegraphics{Projet1_files/figure-latex/unnamed-chunk-2-34.pdf}
\includegraphics{Projet1_files/figure-latex/unnamed-chunk-2-35.pdf}
\includegraphics{Projet1_files/figure-latex/unnamed-chunk-2-36.pdf}
\includegraphics{Projet1_files/figure-latex/unnamed-chunk-2-37.pdf}
\includegraphics{Projet1_files/figure-latex/unnamed-chunk-2-38.pdf}
\includegraphics{Projet1_files/figure-latex/unnamed-chunk-2-39.pdf}
\includegraphics{Projet1_files/figure-latex/unnamed-chunk-2-40.pdf}
\includegraphics{Projet1_files/figure-latex/unnamed-chunk-2-41.pdf}
\includegraphics{Projet1_files/figure-latex/unnamed-chunk-2-42.pdf}
\includegraphics{Projet1_files/figure-latex/unnamed-chunk-2-43.pdf}
\includegraphics{Projet1_files/figure-latex/unnamed-chunk-2-44.pdf}
\includegraphics{Projet1_files/figure-latex/unnamed-chunk-2-45.pdf}
\includegraphics{Projet1_files/figure-latex/unnamed-chunk-2-46.pdf}
\includegraphics{Projet1_files/figure-latex/unnamed-chunk-2-47.pdf}
\includegraphics{Projet1_files/figure-latex/unnamed-chunk-2-48.pdf}
\includegraphics{Projet1_files/figure-latex/unnamed-chunk-2-49.pdf}
\includegraphics{Projet1_files/figure-latex/unnamed-chunk-2-50.pdf}
\includegraphics{Projet1_files/figure-latex/unnamed-chunk-2-51.pdf}
\includegraphics{Projet1_files/figure-latex/unnamed-chunk-2-52.pdf}
\includegraphics{Projet1_files/figure-latex/unnamed-chunk-2-53.pdf}
\includegraphics{Projet1_files/figure-latex/unnamed-chunk-2-54.pdf}
\includegraphics{Projet1_files/figure-latex/unnamed-chunk-2-55.pdf}
\includegraphics{Projet1_files/figure-latex/unnamed-chunk-2-56.pdf}
\includegraphics{Projet1_files/figure-latex/unnamed-chunk-2-57.pdf}
\includegraphics{Projet1_files/figure-latex/unnamed-chunk-2-58.pdf}
\includegraphics{Projet1_files/figure-latex/unnamed-chunk-2-59.pdf}
\includegraphics{Projet1_files/figure-latex/unnamed-chunk-2-60.pdf}
\includegraphics{Projet1_files/figure-latex/unnamed-chunk-2-61.pdf}
\includegraphics{Projet1_files/figure-latex/unnamed-chunk-2-62.pdf}
\includegraphics{Projet1_files/figure-latex/unnamed-chunk-2-63.pdf}
\includegraphics{Projet1_files/figure-latex/unnamed-chunk-2-64.pdf}
\includegraphics{Projet1_files/figure-latex/unnamed-chunk-2-65.pdf}
\includegraphics{Projet1_files/figure-latex/unnamed-chunk-2-66.pdf}

With that, we can use some representations :

\begin{Shaded}
\begin{Highlighting}[]
\CommentTok{\# Refaire le graphe de l\textquotesingle{}article :)}
\end{Highlighting}
\end{Shaded}

\subsection{5. Discussion}\label{discussion}

Comparez vos résultats à ceux du papier. Sont-ils semblables ou
différents? Pourquoi selon vous? Si cela vous semble pertinent, proposez
des pistes d'amélioration de l'analyse.

Well, here, we literally re use the same method as the article
sooooooo\ldots.

maybe using an exponential or a poisson priors ??? like it's more
informative ??? idk

\end{document}
